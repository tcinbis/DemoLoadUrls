% !TEX TS-program = pdflatex
% !TEX encoding = UTF-8 Unicode
\documentclass[11pt]{article} % use larger type; default would be 10pt

\usepackage[utf8]{inputenc} % set input encoding (not needed with XeLaTeX)

%%% Examples of Article customizations
% These packages are optional, depending whether you want the features they provide.
% See the LaTeX Companion or other references for full information.

%%% PAGE DIMENSIONS
\usepackage{geometry} % to change the page dimensions
\geometry{a4paper} % or letterpaper (US) or a5paper or....
\geometry{margin=0.8in} % for example, change the margins to 2 inches all round
% \geometry{landscape} % set up the page for landscape
%   read geometry.pdf for detailed page layout information

\usepackage{graphicx} % support the \includegraphics command and options

\usepackage[ngerman]{babel}
\usepackage{listings}



% \usepackage[parfill]{parskip} % Activate to begin paragraphs with an empty line rather than an indent

%%% PACKAGES
\usepackage{booktabs} % for much better looking tables
\usepackage{array} % for better arrays (eg matrices) in maths
\usepackage{paralist} % very flexible & customisable lists (eg. enumerate/itemize, etc.)
\usepackage{verbatim} % adds environment for commenting out blocks of text & for better verbatim
\usepackage{subfig} % make it possible to include more than one captioned figure/table in a single float
% These packages are all incorporated in the memoir class to one degree or another..

%%% HEADERS & FOOTERS
\usepackage{fancyhdr} % This should be set AFTER setting up the page geometry
\pagestyle{fancy} % options: empty , plain , fancy
\renewcommand{\headrulewidth}{0pt} % customise the layout...
\lhead{}\chead{}\rhead{}
\lfoot{}\cfoot{\thepage}\rfoot{}

%%% SECTION TITLE APPEARANCE
\usepackage{sectsty}
\allsectionsfont{\sffamily\mdseries\upshape} % (See the fntguide.pdf for font help)
% (This matches ConTeXt defaults)

%%% END Article customizations
%%% BEGINN custom customizations
\usepackage[scaled=.90]{helvet}

\renewcommand\familydefault{\sfdefault}

%%% The "real" document content comes below...

\title{\textbf{HACKATHON}}
\author{Max Burgert, Tom Cinbis, Paul Höft und Tim Königl}
%\date{} % Activate to display a given date or no date (if empty),
         % otherwise the current date is printed 

\begin{document}
\maketitle

\section{Aufgabe}
Die Aufgabenstellung für diesen Hackathon lautete, aus ca. zwei Milliarden URLS auf einem Laptop in Echtzeit fünf Vorschläge für eine Auto-Completion zu liefern und ausserdem die Anzahl der möglichen Treffer anzuzeigen. 

\section{Ansatz}
Da die Datenmenge in einer 115 GB grossen .txt Datei gespeichert war, bedurfte es eines Prozesses, diese Datenmenge erst vorzubereiten um anschliessend eine Auto-Completion darauf in angemessener Zeit ausführen zu können.\\
Um mit dieser Datenmenge arbeiten zu können, entschieden wir uns für die Nutzung einer Datenbank. Besonders geeignet hierfür schien uns PostrgeSQL, da diese Datenbank eine optimierte Einlesefunktion grosser Datenmengen in eine Datenbank umfasst.\\
Als Programmiersprache entschieden wir uns für Java, in welcher wir uns am Besten auskannten.

\section{Optimierung}
Eine Indizierung dieser grossen Datenmenge erwies sich in PostgreSQL als unvorteilhaft, da die standardmässige B-tree Indizierung die maximale Bytegrösse überschritten hat und ein Hashen dieser unbrauchbar war.\\
Um besonders die Berechnung für die Anzahl der Treffer deutlich zu beschleunigen, teilten wir die ursprüngliche Datenbank, welche die gesamte Datenmenge umfasste, anhand von \glqq http\grqq, \glqq https\grqq und anschliessend nach \glqq www\grqq, sowie nach den Anfangsbuchstaben \glqq a-z\grqq  auf.\\
Bei entsprechender Eingabe musste so nicht mehr über die gesamte Datenmenge nach passenden Einträgen gesucht werden, sondern nur in der jeweiligen Datenbank.\\
Für Eingaben, welche Fälle beeinhalten bei dem das Aufteilen der Datenbanken keinen Laufzeitvorteil erbrachten, haben wir die Anzahl der Treffer vorberechnet.

\section{Nutzung}
Um den Code testen zu können ist zuerst ein Start des PostgreSQL Servers nötig, eine entsprechende Batchfile zum Starten sowie Beenden ist auf der Festplatte vorhanden. Anschliessend kann mit dem Befehl \begin{lstlisting}
java -jar HackathonGUI-1.0.jar
\end{lstlisting} eine simple GUI gestartet werden.\\
In das Textfeld kann nun Zeichen für Zeichen nach einer beliebigen URL gesucht werden. Unterhalb des Textfeldes werden hierbei fünf mögliche Resultate angezeigt. Die gesamte Anzahl der Treffer ist rechts neben dem Textfeld einzusehen.
Falls die Zeit und Geduld besteht, kann mit dem starten des PostgreSQL Servers und dem Befehl:
\begin{lstlisting}
java -jar HackathonBuild-1.0.jar 'url-sample.txt'
\end{lstlisting}
das Vorbereiten der Daten gestartet werden.

\end{document}
